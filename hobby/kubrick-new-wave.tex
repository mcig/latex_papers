% make title bold

\title{\textbf{Stanley Kubrick and The French New Wave}}
\author{
        RCTV Film Culture Midterm Exam\\
        Mustafa Cig Gokpinar - 19070001001\\
}
\date{\today}

\documentclass[12pt]{article}
\linespread{1.25}

\begin{document}
\maketitle

The French New Wave is one of the art movements that played significant role in the History of Cinema.
It was started by the group of young French filmmakers who were related to the Cahiers du Cinema magazine in the late 1950s and
it was born as a reaction to the classical French cinema and the Hollywood cinema.
\\
The filmmakers wanted to get experimental and they wanted to make films that were different from the classical cinema. However they
lacked the financial resources to make their films. They only had their own ideas and portable cameras that they carried around in the city
and they made their films in the streets. These thinking processes made it possible for them to discover some new techniques like the use of natural light,
the use of long takes, the use of improvisation and the use of real locations. These shooting styles allowed the actors to be more natural and explore the scene
more freely. As a contrast to that the edits were very short and the films were very fast paced and usually the view-time were mostly
around 90 minutes.
\\
% talk about the Auteur theory

\bibliographystyle{abbrv}
\bibliography{main}

\end{document}