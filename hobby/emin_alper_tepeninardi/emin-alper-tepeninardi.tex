% make title bold


\title{\textbf{Stanley Kubrick and The French New Wave}}
\author{
        RCTV Film Culture Midterm Exam\\
        Mustafa Cig Gokpinar - 19070001001\\
}
\date{\today}

\documentclass[12pt]{article}
\linespread{1.25}

\usepackage{graphicx}
\usepackage{url}
\usepackage{placeins}
\usepackage[margin=1in]{geometry}
\graphicspath{ {./images} }

\begin{document}
\maketitle

\section{About the Director}

Emin Alper is a Turkish filmmaker and screenwriter, born in Istanbul in 1974. He received his undergraduate degree in Electrical and Electronic Engineering and later pursued a Master's degree in Film Studies at Istanbul University. Alper has directed and written a number of films, including "Tepenin Ardı" (2012) and "Beyond the Hill" (2012). He has also received various awards for his work, including the Golden Tulip at the Istanbul Film Festival for "Tepenin Ardı" and the Best Screenplay award at the Antalya Film Festival for "Beyond the Hill" (IMDb, 2021).

One common theme in Alper's films is the exploration of social and political issues. In "Tepenin Ardı," for example, the film examines the consequences of the Turkish government's construction of a hydroelectric dam and its impact on the local community (Alper, 2012). Another theme that can be seen in Alper's work is the portrayal of human resilience in the face of adversity. In "Beyond the Hill," for instance, the film tells the story of three generations of a family living in a remote village, and their struggles to maintain their way of life amidst changing societal norms (Alper, 2012).

In terms of character development, Alper often focuses on the portrayal of ordinary people and their struggles in the face of larger societal forces. For example, in "Tepenin Ardı," the film follows the story of a young man named Cemal who is caught in the middle of the conflict between the government and the local community over the construction of the hydroelectric dam (Alper, 2012). Similarly, in "Beyond the Hill," the film focuses on the lives of three generations of a family living in a remote village, and their struggles to maintain their way of life (Alper, 2012).

In terms of cinematography, Alper's films often utilize natural lighting and landscapes to create a sense of authenticity and immersion in the story. In "Tepenin Ardı," for example, the film uses sweeping shots of the natural surroundings to convey the beauty and tranquility of the village before the construction of the dam (Alper, 2012).

% \begin{figure}[h]
%         \centering
%         \includegraphics[width=0.5\linewidth]{tepenin-ardi.jpg}
%         \caption{A still from "Tepenin Ardı," featuring Cemal (played by Ufuk Bayraktar) and his father (played by Necip Memili).}
%         \label{fig:tepenin-ardi}
% \end{figure}

\section{About the Film}

"Tepenin Ardı" is a 2012 Turkish drama film directed and written by Emin Alper. The film tells the story of a young man named Cemal, who is caught in the midst of the conflict between the government and the local community over the construction of a hydroelectric dam in a remote village.

One of the central themes of the film is the impact of development on traditional ways of life. As the government begins construction on the dam, the village is transformed from a peaceful and serene place to one filled with tension and conflict. This theme is reflected in the portrayal of the main character, Cemal, who is torn between his loyalty to his family and his desire to leave the village and pursue his own dreams.

Another theme that can be identified in the film is the struggle between the individual and the collective. As the government and the local community clash over the construction of the dam
, Cemal is forced to confront his own beliefs and values, and decide where his loyalty lies. This theme is exemplified in the relationship between Cemal and his father, who is a vocal opponent of the dam's construction.

In terms of character development, Cemal is the main protagonist of the film. He is a complex and multidimensional character, whose motivations and actions are driven by a combination of personal desires and loyalty to his family and community. Throughout the film, we see Cemal struggle to find his place in the world and make decisions that are true to his values.

Another important character in the film is Cemal's father, who is a strong and proud man who is deeply opposed to the construction of the dam. Despite his determination to protect his community, the film also explores the complexities and flaws of his character, as he grapples with his own personal demons and struggles to maintain his relationships with those around him.

Overall, "Tepenin Ardı" is a thought-provoking and emotionally charged film that explores themes of development, tradition, and the individual versus the collective. Through the portrayal of its complex and multidimensional characters, the film offers a nuanced look at the challenges and struggles faced by ordinary people in the face of larger societal forces.

% \begin{figure}[h]
%         \centering
%         \includegraphics[width=0.5\linewidth]{tepenin-ardi2.jpg}
%         \caption{A still from "Tepenin Ardı," featuring Cemal (played by Ufuk Bayraktar) and his father (played by Necip Memili) in a heated discussion.}
%         \label{fig:tepenin-ardi2}
% \end{figure}

\section{References}

\section*{Bibliography}
% numbered list
\begin{enumerate}
        \item Alper, E. (2012). Tepenin Ardı [Motion picture]. Turkey: Bir Film.
        \item IMDb. (2021). Emin Alper. Retrieved from \url{https://www.imdb.com/name/nm2187371/}
\end{enumerate}
\end{document}
